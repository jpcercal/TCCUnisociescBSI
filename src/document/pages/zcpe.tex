\chapter{O QUE É A ZCPE?}
\label{zcpe}

Conforme afirma \citeonline{dicionarioEscolar}, o termo certificação tem como
significado o ato de certificar ou ainda realizar uma constatação, ou seja, é a
declaração formal emitida por quem possuí credibilidade, enquanto que, a palavra
certificado é um documento que comprova ou fornece a garantia de algo, sendo
que, um exemplo disto é um diploma.

Logo, a \acs{ZCPE}, do inglês, \acl{ZCPE}, é uma prova de certificação oferecida
pela \acs{Zend} que certifica que um profissional está apto a trabalhar com a
tecnologia \acs{PHP} \cite{websiteZendZCPE}.

Entretanto, existe uma segunda prova intitulada como \ac{ZFC},
esta por sua vez, avalia os conhecimentos do profissional no
Framework oficial da \acs{Zend} \cite{websiteZendZFC}.

A \acs{Zend} é uma empresa que atua utilizando a linguagem \acs{PHP} e fornece
a seus clientes soluções rápidas e com alta qualidade, sejam elas:  \textit{web}
ou \textit{mobile}, além disto oferece produtos tais como o \acs{Zend Studio},
\acs{Zend Server} e \acs{Zend Framework}, além de treinamentos e provas de
certificação \cite{websiteZendCompany}.

Em resumo, as provas de certificação são realizadas pela \acs{Pearson VUE}
(\acl{Pearson VUE}), que possuí vários centros credenciados pelo mundo,  exceto
nos seguintes países: Cuba, Coreia do Norte, Sudão e Síria
\cite{websiteZendZFC}. Sendo que, a lista de locais disponíveis para a
realização do exame pode ser encontrada no website desta entidade.

Segundo \citeonline{zendPhp5CertificationStudyGuide}, com a introdução da tão
esperada certificação referente a versão 5 da linguagem \acs{PHP}, o exame se
tornou mais amplo, pois, o conteúdo da prova foi revisado e incrementado,
envolvendo  tópicos adicionais, tais como: orientação à objetos, \textit{web
services} e segurança, portanto, além do profissional ter vivência com a
tecnologia, este por sua vez, necessita de conhecimentos teóricos sólidos
referente a linguagem.

A prova de certificação \acs{Zend} foi projetada tendo como base dois objetivos, são
eles: testar o conhecimento do profissional na tecnologia \acs{PHP} e, o
segundo, fazer com que a prova extraia do profissional o máximo de sua vivência
prática com a tecnologia \cite{theZendPHPCertificationPracticeTestBook}.

Portanto, para \citeonline{theZendPHPCertificationPracticeTestBook}, o
conhecimento na linguagem \acs{PHP} é baseado no seguinte princípio: sua
experiência não deve ser mensurada de acordo com tecnologias
e bibliotecas de terceiros, pois, se isto ocorresse, um profissional que
trabalha a anos com a linguagem e nunca realizou uma conexão com o banco
\acs{MySQL} poderia ser prejudicado na avaliação. Para os autores, a
prova não deve avaliar o seu conhecimento referente a bibliotecas de software
de terceiros, desta forma, o exame aborda questões de funcionalidades da
linguagem  de maneira didática \cite{theZendPHPCertificationPracticeTestBook}.

Logo, o exame irá avaliar a sua habilidade de
entender, interpretar e escrever códigos \acs{PHP} de maneira adequada, então,
o candidato deve estar preparado para ser analizado através de exemplos de
códigos nos quais deverá entender: como funcionam, qual a saída e se há algum \acs{bug},
além disto, algumas questões podem ser complexas, mas, em uma analogia rápida,
um programador passa por isto diariamente ao analisar o código escrito por
outras pessoas para efetuar a correção de um problema
\cite{theZendPHPCertificationPracticeTestBook}.

Conforme \citeonline{theZendPHPCertificationPracticeTestBook}, o candidato que
se submete a prova de certificação \acs{ZCPE} tem noventa minutos para resolver
setenta questões de diferentes áreas do conhecimento com dificuldades variadas
em uma plataforma computacional.

% Aos 20:11 minutos da entrevista:
De acordo com \citeonline{entrevistaAriZCEBrasil}, as perguntas são criadas por
um grupo chamado \textit{Zend PHP Education Advisory Board} que é um grupo formado por
personalidades conhecidas da comunidade \acs{PHP}, são pessoas altamente
gabaritadas, onde alguns são funcionários da \acs{Zend}, enquanto que, outros são autores
reconhecidos,  há ainda desenvolvedores expoentes da área, sendo que, como esses
profissionais possuem uma visão e experiência diferenciada (outros pontos de
vista), as questões acabam sendo heterogêneas, porém, com carater neutro, dando
uma amplitude bastante positiva ao candidato que se submete ao exame.

% Aos 15:38 minutos na entrevista:
Por conseguinte, as questões abordadas na prova envolvem perguntas com
possibilidades de respostas de três tipos, são elas: de campo aberto (intitulada
como \textit{freetext}), de escolha única e de múltipla escolha
\cite{entrevistaAriZCEBrasil}.

% Aos 17:00 minutos na entrevista:
Sendo que, durante a realização da prova, o candidato pode marcar as questões
que geraram dúvidas, para que possam ser revistas pouco antes da finalização do
exame, afim de administrar melhor o tempo fornecido para a conclusão da  prova
\cite{entrevistaAriZCEBrasil}.

De acordo com \citeonline{websiteZendZCPE}, a prova aborda as seguintes
áreas do conhecimento:

\begin{alineas}
    \item \acs{PHP} \textit{basics}: conceitos básicos referentes a
    linguagem, tais como: sintaxe, operadores, variáveis, constantes e
    estrutoras de controle;
    \item \textit{functions}: funções da linguagem, incluindo: sintaxe,
    argumentos, variáveis, retornos e escopo de variáveis;
    \item \textit{data format} \& \textit{types}: formatos e tipos de
    dados, envolvendo: \ac{XML}, \ac{SOAP}, \ac{JSON} e \textit{DateTime};
    \item \textit{web features}: recursos web, tais como: formulários,
    \textit{cookies} e o protocolo \ac{HTTP};
    \item \acs{OOP}: \acl{OOP} ou simplemesmente \ac{POO},
    abordando conceitos como: instanciação, herança, interfaces, exceções e
    constantes;
    \item \textit{security}: segurança, incluindo questões relacionadas com:
    configuração, \ac{XSS}, \textit{\acs{SQL} Injection},
    algoritmos de criptografia e \textit{upload} de arquivos;
    \item \acs{I/O}: \acl{I/O} ou simplemesmente \ac{E/S}, aborda os seguintes
    temas: \textit{streams} e leitura ou escrita de arquivos;
    \item \textit{strings} \& \textit{patterns}: cadeia de caracteres e
    seus padrões, abordando assuntos como por exemplo: expressões
    regulares e substituição ou busca de caracteres.
    \item \textit{databases} \& \acs{SQL}: banco de dados e
    manipulação através de comandos \ac{SQL}, incluindo também: transações e a
    extensão \ac{PDO};
    \item \textit{arrays}: vetores ou matrizes numéricas e
    associativas unidimensionais ou multidimensionais.
\end{alineas}

No que diz respeito ao mercado, como o escopo do \acs{PHP} é muito abrangente,
as grandes empresas precisam ter uma maneira padrão e confiável de avaliar as
habilidades e capacidades de um profissional que atue com a linguagem PHP.
Sendo assim, o principal objetivo da prova é oferecer a empregadores e profissionais
certificados uma forma de avaliação padrão \cite{zendPhp5CertificationStudyGuide}.

% Comando que utilizei para pegar a quantidade de profissionais na página da
% zend: document.getElementById("yellowpagesajaxlist").childNodes.length
Hoje no Brasil, segundo a \citeonline{websiteZendYellowPagesDirectory}, a
quantidade de profissionais certificados na tecnologia \acs{PHP} é relativamente
baixa, se comparado a quantidade de profissionais que trabalham com esta
linguagem diariamente, havendo no Brasil apenas 340 profissionais certificados
desde que a primeira edição da prova entou em vigor (em 2004).

Dentre os motivos para o baixo número de profissionais certificados, acredita-se
que as principais causas sejam: o valor da prova, que perante a
\citeonline{websiteZendPhpCertification} custa \$ 195,00 dólares e, além disto,
os custos com preparações, que podem chegar até os \$ 1000,00 dólares no site da
própria instituição \cite{websiteZendOnlineTraining}.

A fim de facilitar o entendimento, a seguir será apresentado um breve histórico
de todas as versões da prova de certificação na linguagem \acs{PHP} fornecida
pela \acs{Zend}.

\section{HISTÓRICO DA PROVA}

Até o presente momento - em ciclos de três anos - houve quatro edições da
prova, sendo que anteriormente este exame ficou conhecido como \ac{ZCE} e
atualmente denomina-se como \acl{ZCPE} (\acs{ZCPE}). Percebe-se ainda, que a
prova evolui de acordo com a versão da linguagem \acs{PHP}.

A seguir será apresentado de maneira cronológica os logotipos e detalhes
referentes a cada versão do exame que ocorre desde 2004.

\subsection{PHP 4}

Na Figura \ref{fig:logoCertificationPHP4} é apresentada a logo desta edição, que
foi anunciada em 2004 na \ac{OSCON}, sendo que, seu conteúdo baseou-se na versão
4 da linguagem \acs{PHP}, onde, nesse período houveram mais de mil profissionais
certificados em dois anos, pois, este exame expirou em 2006
\cite{entrevistaAriZCEBrasil}.

\begin{figure}[h!tb]
	\caption{ZCE PHP 4}
	\label{fig:logoCertificationPHP4}

	\centering
	\includegraphics[width=0.2\textwidth]{images/logo/php4.png}

	\centering
	\footnotesize Fonte: \citeonline{websiteZendCertifiedEngineerLogos}
\end{figure}

\FloatBarrier 	% Este comando impede que as imagens
				% flutuem a partir deste ponto no seu documento

\subsection{PHP 5}

A edição que abordou a versão 5.1 da tecnologia foi anunciada no ano de
2005 na \textit{php|Works}, mas, efetivou-se apenas em 2007, por conseguinte,
conforme afirma \citeonline{entrevistaAriZCEBrasil}, o exame teve seu conteúdo
revisado e incrementado, em contrapartida, se comparada a edição anterior do
prova, houve um aumento no número de profissionais que obtiveram a nova
certificação, chegando a marca de seis mil certificados. Na Figura
\ref{fig:logoCertificationPHP5} será exibida a logo desta edição.

\begin{figure}[h!tb]
	\caption{ZCE PHP 5}
	\label{fig:logoCertificationPHP5}

	\centering
	\includegraphics[width=0.19\textwidth]{images/logo/php5.png}

	\centering
	\footnotesize Fonte: \citeonline{websiteZendCertifiedEngineerLogos}
\end{figure}

\FloatBarrier 	% Este comando impede que as imagens
				% flutuem a partir deste ponto no seu documento

\subsection{PHP 5.3}

Lançada em 2010, esta edição foi baseada na versão 5.3 da linguagem \acs{PHP},
sendo que, manteve-se disponível até dezembro de 2013, onde, alguns ajustes de
conteúdo foram realizados, tais como: a remoção de temas obsoletos referentes a versão 4
da linguagem e a adição de recursos exclusivos da versão 5.3
\cite{entrevistaAriZCEBrasil}. Na Figura \ref{fig:logoCertificationPHP53} é
apresentada a logo desta edição.

\begin{figure}[h!tb]
	\caption{ZCE PHP 5.3}
	\label{fig:logoCertificationPHP53}

	\centering
	\includegraphics[width=0.19\textwidth]{images/logo/php5-3.png}

	\centering
	\footnotesize Fonte: \citeonline{websiteZendCertifiedEngineerLogos}
\end{figure}

\FloatBarrier 	% Este comando impede que as imagens
				% flutuem a partir deste ponto no seu documento


\subsection{PHP 5.5}

Lançada em 2013 no evento \textit{ZendCon}, esta edição que representa a versão
atual do exame, modificou pela primeira vez o nome da prova de \acs{ZCE} para
\acs{ZCPE}, do inglês, \acl{ZCPE} e aborda a versão atual da tecnologia, o
PHP 5.5 \cite{entrevistaAriZCEBrasil}.

\begin{figure}[h!tb]
	\caption{ZCPE PHP 5.5}
	\label{fig:logoCertificationPHP55}

	\centering
	\includegraphics[width=0.19\textwidth]{images/logo/php5-5.png}

	\centering
	\footnotesize Fonte: \citeonline{websiteZendCertifiedEngineerLogos}
\end{figure}

\FloatBarrier 	% Este comando impede que as imagens
				% flutuem a partir deste ponto no seu documento

