\chapter{ORIENTAÇÃO A OBJETOS}
\label{orientacaoAObjetos}

A orientação a objetos traz uma nova forma de se pensar a construção de sistemas
computacionais. Pois, diferentemente do pensamento em que se tinha durante a 
programação estruturada, onde eram definidos pequenos trechos de código sem que 
houvesse um contexto para agrupá-los, este novo paradigma faz com que os sistemas 
sejam construídos de maneira organizada, uma vez que trabalhamos com estruturas 
semelhantes as que conhecemos do nosso dia a dia
\cite{phpProgramandoComOrientacaoAObjetos}.

A orientação a objetos nos permite associar valores e funções em uma única
unidade: o objeto. Ao invés de termos variáveis com prefixos que indiquem o 
motivo de sua existência, ou valores armazenados em matrizes para manter os 
elementos juntos, com o uso de objetos podemos adicionar funcionalidades e 
comportamentos a uma unidade de software criando um novo tipo de dados: as 
classes \cite{phpMasterWriteCuttingEdgeCode}.

\subsection{Classe}

A classe é uma estrutura que agrupa as propriedades e os métodos de forma
abstrata com base no modelo de negócio do software que será desenvolvido.
Sendo assim, é uma estrutura estática que agrupa de maneira lógica e descreve
propriedades e métodos com base no modelo de negócio representando uma abstração
da realidade \cite{phpProgramandoComOrientacaoAObjetos}.

Por conta disto, uma classe pode ser considerada como um modelo ou
\textit{template}, no qual, com base nesses modelos podem ser criados vários objetos.

Em uma analogia com o processo de preparação de um bolo, temos a classe como
sendo a receita e o objeto como o bolo. Onde, com base em uma receita podemos fazer
vários bolos.

Para \citeonline{c++Absoluto}, uma classe é um novo tipo de dados assim como os já
existentes tipos primitivos: \textit{int} e \textit{double}, portanto, um método
ou uma variável podem usar uma classe ao trocarem mensagens, sendo assim a classe
poderia ser utilizada como dado de entrada ou saída.

Portanto, as classes são os blocos de construção de uma aplicação, que quando
unidos dão origem a um software \cite{learningJava}. Sendo que, pode ser
composta por: métodos, propriedades, códigos de construção e destruição,
utilizando conceitos de: herança, polimorfismo, encapsulamento, interfaces e
exceções. E por fim, um conjunto de classes poderá ser agrupado em um pacote.

\begin{figure}[h!tb]
	\caption{Definição de uma Classe na linguagem PHP}
	\label{fig:classe}

	\centering
	\includegraphics[width=0.75\textwidth]{images/class.png}

	\centering
	\footnotesize Fonte: \fonteOAutor
\end{figure}

\FloatBarrier 	% Este comando impede que as imagens
				% flutuem a partir deste ponto no seu documento

Logo abaixo será apresentada a análise do código exibido na
Figura \ref{fig:classe}:

\begin{enumerate}[a)]
    \item linha 1: temos o início da execução de um bloco de código
    PHP;
    \item linha 3: informamos ao interpretador da linguagem PHP que
    estamos definindo um novo tipo de dados (uma nova estrutura) que será
    identificada através do nome \textit{Carro};
    \item linha 4: o símbolo \textbf{\{} se refere a abertura de um
    bloco de código, ou seja, informa ao interpretador onde inicia-se a definição de
    características ou dados (propriedades ou atributos) e ações (métodos).
    Ambos os conceitos métodos e propriedades veremos ao decorrer deste
    capítulo;
    \item linha 6: o símbolo \textbf{\}} se refere ao fechamento de um
    bloco de código, ou seja, informa ao interpretador onde terminam as
    definições de propriedades e métodos.
\end{enumerate}
\section{OBJETO}

Um objeto é uma estrutura dinâmica criada com base em uma classe. Sendo que, 
com base em uma classe podemos ter vários objetos, cada qual com suas 
propriedades \cite{phpProgramandoComOrientacaoAObjetos}. 

De forma breve, os objetos são as instâncias de uma classe. Sendo que, as
classes existem somente no código fonte de uma aplicação, enquanto que, as 
instâncias de uma classe existem durante a execução de um programa. Portanto, 
o software poderá criar vários objetos sob demanda tendo como base um mesmo 
modelo \cite{ios7ProgrammingFundamentalsObjectiveCXcodeAndCocoaBasics}. Deste
modo, esses objetos são criados (instanciados) através de métodos construtores 
e destruídos (eliminados) através de métodos destrutores em tempo de execução
\cite{umlEC++GuiaPraticoDeDesenvolvimentoOrientadoAObjeto}. Veremos no decorrer
deste capítulo como funcionam os métodos construtores e destrutores.

\begin{figure}[h!tb]
	\label{fig:objeto}
	
	\centering
	\caption{Criação de um Objeto na linguagem PHP}
	
	\centering
	\includegraphics[width=0.75\textwidth]{images/object.png}
	
	\centering
	\footnotesize Fonte: \fonteOAutor
\end{figure}

\FloatBarrier 	% Este comando impede que as imagens 
				% flutuem a partir deste ponto no seu documento

A seguir, será apresentada a análise do código exibido na
figura \ref{fig:objeto}: 

\begin{enumerate}[a)]
    \item \textbf{linha 1:} temos o início da execução de um bloco de código PHP;
    \item \textbf{linha 3:} criamos uma variável chamada \textbf{\$carro};
    chamamos um operador de atribuição da linguagem (representado pelo símbolo
    \textbf{=}) que irá atribuir o valor que está à direita do operador na
    variável que está à esquerda; em seguida informamos um outro operador da 
    linguagem (representado pelo símbolo \textbf{new}) que é responsável por
    criar uma referência em memória para o nosso tipo de dados que está sendo 
    criado e, por fim, dizemos que classe queremos instanciar, que neste caso, 
    chama-se Carro. Logo após, o símbolo \textbf{()} representa um método
    construtor (conceito que veremos ainda neste capítulo de Orientação a Objetos). 
    Uma informação importante é que toda instrução da linguagem PHP termina 
    com o símbolo de ponto-e-vírgula.
\end{enumerate}
\subsection{Método}

Segundo \citeonline{php5ConceitosProgramacaoEIntegracaoComBancoDeDados}, um
método pode ser definido como sendo as operações que manipulam os dados de uma
classe, ou seja, definem o que as classes podem e sabem fazer, como por exemplo
acelerar um carro modificando o valor de sua propriedade chamada
\textit{velocidade} para um valor crescente em um determinado espaço de tempo.

Os métodos também podem ser chamados de funções membro \cite{c++ComoProgramar}.

Se comparado a programação estruturada um método pode ser considerado como sendo
uma função que está associada a uma classe \cite{programmingPhp}.

\begin{figure}[h!tb]
	\caption{Criação de um Método utilizando a linguagem PHP}
	\label{fig:metodo}

	\centering
	\includegraphics[width=0.75\textwidth]{images/method.png}

	\centering
	\footnotesize Fonte: \fonteOAutor
\end{figure}

\FloatBarrier 	% Este comando impede que as imagens
				% flutuem a partir deste ponto no seu documento

Na sequência, você irá conferir uma explicação referente ao código que foi
apresentado na Figura \ref{fig:metodo}:

\begin{enumerate}[a)]
    \item linha 1: vê-se o início da execução de um bloco de código PHP;
    \item linha 3: define-se uma classe chamada \textit{Carro};
    \item linha 4: informa-se onde inicia o bloco de uma classe;
    \item linha 5 e 7: cria-se duas propriedades para a classe
    \textit{Carro}, são elas: \textit{\$cor} e \textit{\$quantidadePortas};
    \item linha 9: é solicitado para que o interpretador crie um método
    cuja visibilidade seja pública e definisse que este método será identificado
    pelo nome \textit{setCor}.
    Além disso, informasse que este método deve receber um parâmetro (um valor
    que  irá configurar a propriedade de uma classe);
    \item linha 10: define-se onde inicia o bloco cujo escopo
    corresponda ao método \textit{setCor};
    \item linha 11: utilizou-se uma variável especial chamada
    \textit{\$this}, esta variável permite acessar qualquer propriedade ou
    método dentro da própria classe ou \textit{superclasse}; depois, usou-se
    o operador de acesso a um objeto (representado pelo símbolo \textbf{->}); em
    seguida, informa-se ao interpretado do \acs{PHP}, a necessidade de manipular
    o valor da propriedade \textit{cor}, sendo que, ela deverá receber o valor
    informado como parâmetro para o método \textit{setCor};
    \item linha 12: define-se onde termina o bloco que corresponde ao
    método \textit{setCor};
    \item linha 13: informa-se o encerramento do bloco de uma classe.
\end{enumerate}

Isto permite que de fora da classe carro outro objeto configure a cor de um
veículo passando uma mensagem ao objeto carro com a cor solicitada. Na Figura
\ref{fig:chamadaMetodo} será exibido um exemplo desta situação:


\begin{figure}[h!tb]
	\caption{Chamada de um Método utilizando a linguagem PHP}
	\label{fig:chamadaMetodo}

	\centering
	\includegraphics[width=0.75\textwidth]{images/method-call.png}

	\centering
	\footnotesize Fonte: \fonteOAutor
\end{figure}

\FloatBarrier 	% Este comando impede que as imagens
				% flutuem a partir deste ponto no seu documento

Abaixo, você irá conferir uma explicação referente ao código que foi
apresentado na Figura \ref{fig:chamadaMetodo}:

\begin{enumerate}[a)]
    \item linha 1: tem-se o início da execução de um bloco de código
    PHP;
    \item linha 2: ocorre a criação de um objeto do tipo \textit{Carro};
    \item linha 5: realiza-se a chamada de um método chamado
    \textit{setCor}, sendo que, informa-se um parâmetro para ele, que na
    linguagem PHP representa uma \textit{string} (cadeira de caracteres). Logo,
    configura-se o valor da propriedade \textit{cor} da classe \textit{Carro}
    para receber o valor \textit{azul}.
\end{enumerate}
\section{PROPRIEDADE}
Assim como a comparação com a programação estruturada abordando o que é um 
método, seguindo a mesma analogia, uma propriedade (também conhecida como 
atributo, variável membro ou ainda variável de instância) pode ser considerada 
como os dados que um objeto possuí, descrevendo desta forma, as características 
que a ele pertencem \cite{programmingPhp}.

Sendo assim, os atributos são variáveis que estão definidas dentro de uma
classe, deste modo, geralmente são acessados através de uma interface de acesso,
pois não estão visíveis para que outros objetos manipulem os dados diretamente, 
se isto ocorresse, poderia comprometer a segurança da informação e também o 
conceito de que cada objeto possuí uma finalidade.

Então, uma propriedade (pensando na classe carro) poderia ser uma característica
que um carro possuí no mundo real. Portanto, poderíamos levantar de acordo com 
o nosso conhecimento rapidamente os seguintes parâmetros que definem um carro: 
cor, quantidade de portas, possuí direção hidráulica, etc.


\begin{figure}[h!tb]
	\label{fig:propriedade}
	
	\centering
	\caption{Criação de duas propriedade para a Classe Carro implementadas na
	linguagem PHP}
	
	\centering
	\includegraphics[width=0.75\textwidth]{images/property.png}
	
	\centering
	\footnotesize Fonte: \fonteOAutor
\end{figure}

\FloatBarrier 	% Este comando impede que as imagens 
				% flutuem a partir deste ponto no seu documento

Por conseguinte, veremos abaixo uma análise da figura \ref{fig:propriedade}:

\begin{enumerate}[a)]
    \item \textbf{linha 1:} temos o início da execução de um bloco de código PHP;
    \item \textbf{linha 3:} definimos uma classe chamada \textit{Carro};
    \item \textbf{linha 5:} definimos uma palavra reservada chamada
    \textit{private} que se refere a visibilidade da propriedade no contexto  de
    um conjunto de objetos (veremos mais detalhes sobre a visibilidade adiante) 
    e, em seguida, é definido o nome de uma variável, que neste caso chama-se
    \textit{\$cor};
    \item \textbf{linha 7:} é definida uma segunda propriedade para a classe
    carro chamada de \textit{\$quantidadePortas};
    \item \textbf{linha 8:} definimos onde termina o bloco que compreende a
    classe \textit{Carro}.
\end{enumerate}

\subsection{Propriedade Estática}

Como vimos anteriormente, as classes são formadas por variáveis de instância e
métodos. Entretanto, as variáveis que forem declaradas com a palavra reservada 
static, serão compartilhadas por toda a classe. Por conta disto, as variáveis 
que assim forem definidas, recebem o nome de variáveis estáticas ou ainda 
propriedades estáticas \cite{learningJava}.

Veremos abaixo um exemplo de propriedade estática utilizando a linguagem PHP:
