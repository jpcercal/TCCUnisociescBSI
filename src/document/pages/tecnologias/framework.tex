\section{FRAMEWORK}

Os \textit{frameworks} surgiram com o objetivo de facilitar o desenvolvimento de
software, pois utilizando-o, programadores focam os seus esforços no que é
realmente é importante, as regras de negócio do sistema.

Conforme afirma \citeonline{padroesDeProjetosSolucoesReutilizaveis}, os
\textit{frameworks} ditam a arquitetura de uma aplicação, eles cooperam afim de fornecer 
uma abstração de tarefas que são comuns entre diferentes sistemas, sendo que, 
através da combinação de padrões de projetos é possível que ele possa ser 
reutilizado em diferentes projetos de maneira transparente permitindo que um 
projetista possa se concentar nos aspectos específicos da aplicação.

Para \citeonline{frameworksParaDesenvolvimentoEmPHP}, um dos motivos para o 
emprego de \textit{frameworks} nos projetos de software está no fato de minimizar as 
tarefas repetitivas, permitindo que uma equipe de desenvolvimento entregue 
soluções de maneira mais rápida e com uma qualidade superior.

A seguir, serão apresentados dois \textit{frameworks} que são referência no mercado
quando tratamos da linguagem \acs{PHP}, sendo ambos de código livre, são
eles: \textit{Symfony 2} e \textit{Zend 2}.

\subsection{Symfony 2}


\subsection{Zend 2}

O \ac{ZF2} é a última atualização do já conhecido \textit{Zend
Framework}, foi e continua sendo desenvolvido pela \acs{Zend}, sendo que, nesta
nova versão, o processo para criação de aplicações web complexas está ainda mais
simplificado tudo isto graças a utilização de componentes fracamente acoplados.

Sendo assim, para \citeonline{zendFramework2ByExampleBeginnersGuide}, o
\acs{ZF2} fornece uma estrutura altamente robusta e escalável para o 
desenvolvimento de aplicações web.

Conforme define
\citeonline{criandoAplicacoesPHPComZendEDojoPadroesEReusoComFrameworks}, o 
\textit{Zend Framework} encapsula a experiência do desenvolvimento de aplicações
que utilizam a linguagem \acs{PHP} graças a sua biblioteca de componentes reutilizáveis.

Segundo \citeonline{zendFramework2ByExampleBeginnersGuide}, dentre as novidades
desta nova versão se comparada a primeira \textit{release} do \textit{framework}
estão:

\begin{enumerate}[a)]
    \item suporte aos recursos de namespaces e closures do \acs{PHP} 5.3;
    \item desenvolvimento de aplicações com arquitetura modular;
    \item suporte a gerenciador de eventos;
    \item suporte a injeção de dependência.
\end{enumerate}