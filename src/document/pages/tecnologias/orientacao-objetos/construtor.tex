\subsection{Construtor}

Como vimos anteriormente através de exemplos utilizando linguagem de
programação PHP, os objetos são criados (instanciados) utilizando-se um
operador especial, este operador chama-se \textit{new}.

Segundo \citeonline{learningJava}, um construtor é um método especial que tem
como responsabilidade inicializar as propriedades de uma classe. Sendo que,
diferentemente de outras linguagens, como por exemplo Java que nomeia o  método
construtor com o mesmo nome da classe, a linguagem de script PHP define  que um
método construtor deve chamar-se \textit{\_\_construct}.

Sabendo disto, toda vez que o interpretador encontrar a palavra reservada
\textit{new} este irá executar o método construtor da classe que está sendo
instanciada.

Caso você defina uma classe e não informe explicitamente um método construtor, a
linguagem em tempo de execução irá inicializar as propriedades com valores nulos.

Além disto, os métodos construtores - da mesma forma que métodos convencionais –
podem aceitar parâmetros de inicialização \cite{learningJava}.

Portanto, os métodos construtores são peças-chave para permitir a configuração
inicial de um objeto.

Veremos abaixo um exemplo de método construtor utilizando a linguagem PHP:

Uma vez que conhecemos os conceitos de métodos construtores, veremos a seguir  o
conceito de métodos destrutores.