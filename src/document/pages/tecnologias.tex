\chapter{ARQUITETURA E TECNOLOGIAS}
\label{tecnologias}
 
Para \citeonline{engenhariaDeSoftware}, a arquitetura de software descreve a
estrutura de um sistema computacional e como eles se relacionam, sendo assim 
além de serem consideradas propriedades externas, também existe a definição de
unidades menores de uma aplicação, nomeadas como componentes de software.

Conforme afirmam
\citeonline{artigoASociedadeEmRedeDoConhecimentoAAccaoPolitica}, a  \ac{TI} é um
conjunto de tecnologias envolvendo \textit{software} e \textit{hardware}, 
possibilitando o manuseio  de informações.

Sendo assim, um sistema de informação pode ser considerado como um conjunto de
componentes de software inter-relacionados, que são desenvolvidos em pequenas 
partes a fim de abstrair e solucionar um problema do mundo real de maneira 
simplificada.
 
Neste capítulo, serão apresentados: uma breve introdução da linguagem \acs{PHP};
conceitos de orientação a objetos com exemplos empregados em \acs{PHP}; os 
dois bancos de dados livres mais populares para projetos web: o
\acs{MySQL} e o \acs{PostgreSQL}; tecnologias para desenvolvimento de aplicações
web, tais como: \acs{HTML}, \acs{CSS} e \acs{JS} e conceitos relacionados a
\textit{frameworks} que descrevem a arquitetura de uma aplicação orientada a
objetos.

\usepackage{minted}

\renewcommand{\theFancyVerbLine}
{
	\sffamily \textcolor[rgb]{0.05,0.05,0.05}
	{\footnotesize \oldstylenums{
		\arabic{FancyVerbLine}
	}}
}

% Exemplo de Uso:
% 
% \inputminted[
% 		mathescape,
%   	linenos,
%   	numbersep=5pt,
%   	gobble=0,
%   	frame=lines,
%   	framesep=10mm
% ]{php}{images/php/method-call.php}
% \label{fig:classe2}
% \centering
% \footnotesize Fonte: \fonteOAutor
\section{ORIENTAÇÃO A OBJETOS}

A orientação a objetos traz uma nova forma de se pensar a construção de sistemas
computacionais. Pois, diferentemente do pensamento em que se tinha durante a 
programação estruturada, onde eram definidos pequenos trechos de código sem que 
houvesse um contexto para agrupá-los, este novo paradigma faz com que os sistemas 
sejam construídos de maneira organizada, uma vez que trabalha-se com estruturas 
semelhantes as que são conhecidas no nosso dia a dia
\cite{phpProgramandoComOrientacaoAObjetos}.

A orientação a objetos nos permite associar valores e funções em uma única
unidade: o objeto. Ao invés de definir variáveis com prefixos que indiquem o 
motivo de sua existência, ou valores armazenados em matrizes para manter os 
elementos juntos, com o uso de objetos é possível adicionar funcionalidades e 
comportamentos a uma unidade de software criando um novo tipo de dados: as 
classes \cite{phpMasterWriteCuttingEdgeCode}.

Na sequência, serão apresentados alguns conceitos da \acl{POO} (\acs{POO}), do
inglês, \ac{OOP}, com exemplos de implementação na linguagem \acs{PHP}.

\subsection{Classe}

A classe é uma estrutura que agrupa as propriedades e os métodos de forma
abstrata com base no modelo de negócio do software que será desenvolvido.
Sendo assim, é uma estrutura estática que agrupa de maneira lógica e descreve
propriedades e métodos com base no modelo de negócio representando uma abstração
da realidade \cite{phpProgramandoComOrientacaoAObjetos}.

Por conta disto, uma classe pode ser considerada como um modelo ou
\textit{template}, no qual, com base nesses modelos podem ser criados vários objetos.

Em uma analogia com o processo de preparação de um bolo, temos a classe como
sendo a receita e o objeto como o bolo. Onde, com base em uma receita podemos fazer
vários bolos.

Para \citeonline{c++Absoluto}, uma classe é um novo tipo de dados assim como os já
existentes tipos primitivos: \textit{int} e \textit{double}, portanto, um método
ou uma variável podem usar uma classe ao trocarem mensagens, sendo assim a classe
poderia ser utilizada como dado de entrada ou saída.

Portanto, as classes são os blocos de construção de uma aplicação, que quando
unidos dão origem a um software \cite{learningJava}. Sendo que, pode ser
composta por: métodos, propriedades, códigos de construção e destruição,
utilizando conceitos de: herança, polimorfismo, encapsulamento, interfaces e
exceções. E por fim, um conjunto de classes poderá ser agrupado em um pacote.

\begin{figure}[h!tb]
	\caption{Definição de uma Classe na linguagem PHP}
	\label{fig:classe}

	\centering
	\includegraphics[width=0.75\textwidth]{images/class.png}

	\centering
	\footnotesize Fonte: \fonteOAutor
\end{figure}

\FloatBarrier 	% Este comando impede que as imagens
				% flutuem a partir deste ponto no seu documento

Logo abaixo será apresentada a análise do código exibido na
Figura \ref{fig:classe}:

\begin{enumerate}[a)]
    \item linha 1: temos o início da execução de um bloco de código
    PHP;
    \item linha 3: informamos ao interpretador da linguagem PHP que
    estamos definindo um novo tipo de dados (uma nova estrutura) que será
    identificada através do nome \textit{Carro};
    \item linha 4: o símbolo \textbf{\{} se refere a abertura de um
    bloco de código, ou seja, informa ao interpretador onde inicia-se a definição de
    características ou dados (propriedades ou atributos) e ações (métodos).
    Ambos os conceitos métodos e propriedades veremos ao decorrer deste
    capítulo;
    \item linha 6: o símbolo \textbf{\}} se refere ao fechamento de um
    bloco de código, ou seja, informa ao interpretador onde terminam as
    definições de propriedades e métodos.
\end{enumerate}
\section{OBJETO}

Um objeto é uma estrutura dinâmica criada com base em uma classe. Sendo que, 
com base em uma classe podemos ter vários objetos, cada qual com suas 
propriedades \cite{phpProgramandoComOrientacaoAObjetos}. 

De forma breve, os objetos são as instâncias de uma classe. Sendo que, as
classes existem somente no código fonte de uma aplicação, enquanto que, as 
instâncias de uma classe existem durante a execução de um programa. Portanto, 
o software poderá criar vários objetos sob demanda tendo como base um mesmo 
modelo \cite{ios7ProgrammingFundamentalsObjectiveCXcodeAndCocoaBasics}. Deste
modo, esses objetos são criados (instanciados) através de métodos construtores 
e destruídos (eliminados) através de métodos destrutores em tempo de execução
\cite{umlEC++GuiaPraticoDeDesenvolvimentoOrientadoAObjeto}. Veremos no decorrer
deste capítulo como funcionam os métodos construtores e destrutores.

\begin{figure}[h!tb]
	\label{fig:objeto}
	
	\centering
	\caption{Criação de um Objeto na linguagem PHP}
	
	\centering
	\includegraphics[width=0.75\textwidth]{images/object.png}
	
	\centering
	\footnotesize Fonte: \fonteOAutor
\end{figure}

\FloatBarrier 	% Este comando impede que as imagens 
				% flutuem a partir deste ponto no seu documento

A seguir, será apresentada a análise do código exibido na
figura \ref{fig:objeto}: 

\begin{enumerate}[a)]
    \item \textbf{linha 1:} temos o início da execução de um bloco de código PHP;
    \item \textbf{linha 3:} criamos uma variável chamada \textbf{\$carro};
    chamamos um operador de atribuição da linguagem (representado pelo símbolo
    \textbf{=}) que irá atribuir o valor que está à direita do operador na
    variável que está à esquerda; em seguida informamos um outro operador da 
    linguagem (representado pelo símbolo \textbf{new}) que é responsável por
    criar uma referência em memória para o nosso tipo de dados que está sendo 
    criado e, por fim, dizemos que classe queremos instanciar, que neste caso, 
    chama-se Carro. Logo após, o símbolo \textbf{()} representa um método
    construtor (conceito que veremos ainda neste capítulo de Orientação a Objetos). 
    Uma informação importante é que toda instrução da linguagem PHP termina 
    com o símbolo de ponto-e-vírgula.
\end{enumerate}
\subsection{Método}

Segundo \citeonline{php5ConceitosProgramacaoEIntegracaoComBancoDeDados}, um
método pode ser definido como sendo as operações que manipulam os dados de uma
classe, ou seja, definem o que as classes podem e sabem fazer, como por exemplo
acelerar um carro modificando o valor de sua propriedade chamada
\textit{velocidade} para um valor crescente em um determinado espaço de tempo.

Os métodos também podem ser chamados de funções membro \cite{c++ComoProgramar}.

Se comparado a programação estruturada um método pode ser considerado como sendo
uma função que está associada a uma classe \cite{programmingPhp}.

\begin{figure}[h!tb]
	\caption{Criação de um Método utilizando a linguagem PHP}
	\label{fig:metodo}

	\centering
	\includegraphics[width=0.75\textwidth]{images/method.png}

	\centering
	\footnotesize Fonte: \fonteOAutor
\end{figure}

\FloatBarrier 	% Este comando impede que as imagens
				% flutuem a partir deste ponto no seu documento

Na sequência, você irá conferir uma explicação referente ao código que foi
apresentado na Figura \ref{fig:metodo}:

\begin{enumerate}[a)]
    \item linha 1: vê-se o início da execução de um bloco de código PHP;
    \item linha 3: define-se uma classe chamada \textit{Carro};
    \item linha 4: informa-se onde inicia o bloco de uma classe;
    \item linha 5 e 7: cria-se duas propriedades para a classe
    \textit{Carro}, são elas: \textit{\$cor} e \textit{\$quantidadePortas};
    \item linha 9: é solicitado para que o interpretador crie um método
    cuja visibilidade seja pública e definisse que este método será identificado
    pelo nome \textit{setCor}.
    Além disso, informasse que este método deve receber um parâmetro (um valor
    que  irá configurar a propriedade de uma classe);
    \item linha 10: define-se onde inicia o bloco cujo escopo
    corresponda ao método \textit{setCor};
    \item linha 11: utilizou-se uma variável especial chamada
    \textit{\$this}, esta variável permite acessar qualquer propriedade ou
    método dentro da própria classe ou \textit{superclasse}; depois, usou-se
    o operador de acesso a um objeto (representado pelo símbolo \textbf{->}); em
    seguida, informa-se ao interpretado do \acs{PHP}, a necessidade de manipular
    o valor da propriedade \textit{cor}, sendo que, ela deverá receber o valor
    informado como parâmetro para o método \textit{setCor};
    \item linha 12: define-se onde termina o bloco que corresponde ao
    método \textit{setCor};
    \item linha 13: informa-se o encerramento do bloco de uma classe.
\end{enumerate}

Isto permite que de fora da classe carro outro objeto configure a cor de um
veículo passando uma mensagem ao objeto carro com a cor solicitada. Na Figura
\ref{fig:chamadaMetodo} será exibido um exemplo desta situação:


\begin{figure}[h!tb]
	\caption{Chamada de um Método utilizando a linguagem PHP}
	\label{fig:chamadaMetodo}

	\centering
	\includegraphics[width=0.75\textwidth]{images/method-call.png}

	\centering
	\footnotesize Fonte: \fonteOAutor
\end{figure}

\FloatBarrier 	% Este comando impede que as imagens
				% flutuem a partir deste ponto no seu documento

Abaixo, você irá conferir uma explicação referente ao código que foi
apresentado na Figura \ref{fig:chamadaMetodo}:

\begin{enumerate}[a)]
    \item linha 1: tem-se o início da execução de um bloco de código
    PHP;
    \item linha 2: ocorre a criação de um objeto do tipo \textit{Carro};
    \item linha 5: realiza-se a chamada de um método chamado
    \textit{setCor}, sendo que, informa-se um parâmetro para ele, que na
    linguagem PHP representa uma \textit{string} (cadeira de caracteres). Logo,
    configura-se o valor da propriedade \textit{cor} da classe \textit{Carro}
    para receber o valor \textit{azul}.
\end{enumerate}
\section{PROPRIEDADE}
Assim como a comparação com a programação estruturada abordando o que é um 
método, seguindo a mesma analogia, uma propriedade (também conhecida como 
atributo, variável membro ou ainda variável de instância) pode ser considerada 
como os dados que um objeto possuí, descrevendo desta forma, as características 
que a ele pertencem \cite{programmingPhp}.

Sendo assim, os atributos são variáveis que estão definidas dentro de uma
classe, deste modo, geralmente são acessados através de uma interface de acesso,
pois não estão visíveis para que outros objetos manipulem os dados diretamente, 
se isto ocorresse, poderia comprometer a segurança da informação e também o 
conceito de que cada objeto possuí uma finalidade.

Então, uma propriedade (pensando na classe carro) poderia ser uma característica
que um carro possuí no mundo real. Portanto, poderíamos levantar de acordo com 
o nosso conhecimento rapidamente os seguintes parâmetros que definem um carro: 
cor, quantidade de portas, possuí direção hidráulica, etc.


\begin{figure}[h!tb]
	\label{fig:propriedade}
	
	\centering
	\caption{Criação de duas propriedade para a Classe Carro implementadas na
	linguagem PHP}
	
	\centering
	\includegraphics[width=0.75\textwidth]{images/property.png}
	
	\centering
	\footnotesize Fonte: \fonteOAutor
\end{figure}

\FloatBarrier 	% Este comando impede que as imagens 
				% flutuem a partir deste ponto no seu documento

Por conseguinte, veremos abaixo uma análise da figura \ref{fig:propriedade}:

\begin{enumerate}[a)]
    \item \textbf{linha 1:} temos o início da execução de um bloco de código PHP;
    \item \textbf{linha 3:} definimos uma classe chamada \textit{Carro};
    \item \textbf{linha 5:} definimos uma palavra reservada chamada
    \textit{private} que se refere a visibilidade da propriedade no contexto  de
    um conjunto de objetos (veremos mais detalhes sobre a visibilidade adiante) 
    e, em seguida, é definido o nome de uma variável, que neste caso chama-se
    \textit{\$cor};
    \item \textbf{linha 7:} é definida uma segunda propriedade para a classe
    carro chamada de \textit{\$quantidadePortas};
    \item \textbf{linha 8:} definimos onde termina o bloco que compreende a
    classe \textit{Carro}.
\end{enumerate}

\subsection{Propriedade Estática}

Como vimos anteriormente, as classes são formadas por variáveis de instância e
métodos. Entretanto, as variáveis que forem declaradas com a palavra reservada 
static, serão compartilhadas por toda a classe. Por conta disto, as variáveis 
que assim forem definidas, recebem o nome de variáveis estáticas ou ainda 
propriedades estáticas \cite{learningJava}.

Veremos abaixo um exemplo de propriedade estática utilizando a linguagem PHP:

\subsection{Constante}

Assim como as constantes globais utilizadas na linguagem de programação
\acs{PHP} através da função \textit{define}, esta tecnologia também dispõem de
uma maneira que permite definirmos uma constante em uma classe ou em uma
interface. Da mesma maneira que as propriedades estáticas as constantes podem
ser acessadas diretamente de dentro do escopo da classe utilizando-se de um
operador especial denominado \textit{self}, ou ainda, no caso da constante, ser
acessada de fora do escopo da classe através do nome da classe \cite{programmingPhp}.

Quando uma constante de uma classe é definida - da mesma maneira que uma
constante global -  seu valor não poderá ser alterado no decorrer da vida útil
da aplicação. Uma prática comumente utilizada pelos desenvolvedores de software
é definir o nome de uma constante com caixa alta, isto permite que ela seja
identificada rapidamente em um trecho de código. Por conseguinte, uma constante
é um identificador que recebe apenas um valor de inicialização, por conta disto,
seu valor não poderá ser alterado durante a execução do aplicativo.

A seguir na Figura \ref{fig:constante}, é apresentado um exemplo de implementação
de uma constante utilizando a linguagem de programação \acs{PHP}:

\begin{figure}[h!tb]
	\caption{Constante declarada na linguagem PHP}
	\label{fig:constante}

	\centering
	\includegraphics[width=0.75\textwidth]{images/const.png}

	\centering
	\footnotesize Fonte: \fonteOAutor
\end{figure}

\FloatBarrier 	% Este comando impede que as imagens
				% flutuem a partir deste ponto no seu documento

Na sequência, apresenta-se em detalhes as linhas de código exibidas na Figura
\ref{fig:constante}:

\begin{alineas}
    \item linha 1: vê-se o início da execução de um bloco de código PHP;
    \item linha 3: identifica-se a declaração da classe \textit{Pneu};
    \item linha 5: define-se a constante que representa um pneu normal,
    atribuindo a ela o valor numérico 1, que pode ser um código de
    identificação;
    \item linha 7: informa-se outra constante relativa a outro tipo de pneu,
    neste caso, o de alta performance.
\end{alineas}

Logo a seguir, exibe-se o conceito de mensagem que permite que os
objetos se comuniquem entre si.

\section{MENSAGEM}

Um software que foi desenvolvido utilizando os conceitos da orientação a 
objetos tem sua execução através da comunicação entre os diversos componentes 
de software, estes componentes chamados de objetos trocam mensagens com o 
objetivo de realizar uma tarefa, isto se faz necessário porque cada objeto  tem
uma responsabilidade para o qual foi projetado na fase de análise.

Assim sendo, uma mensagem é um pedido para que um objeto execute uma ação 
através da chamada de um método, sendo que, este pode alterar o estado de 
outros objetos afim de completar a sua tarefa e, quando ele finalmente termina
a sua execução, geralmente notifica quem solicitou a execução do serviço, ou
seja, retorna algum valor para o objeto que solicitou a operação 
\cite{c++ComoProgramar}.
\subsection{Construtor}

Como vimos anteriormente através de exemplos utilizando linguagem de
programação PHP, os objetos são criados (instanciados) utilizando-se um
operador especial, este operador chama-se \textit{new}.

Segundo \citeonline{learningJava}, um construtor é um método especial que tem
como responsabilidade inicializar as propriedades de uma classe. Sendo que,
diferentemente de outras linguagens, como por exemplo Java que nomeia o  método
construtor com o mesmo nome da classe, a linguagem de script PHP define  que um
método construtor deve chamar-se \textit{\_\_construct}.

Sabendo disto, toda vez que o interpretador encontrar a palavra reservada
\textit{new} este irá executar o método construtor da classe que está sendo
instanciada.

Caso você defina uma classe e não informe explicitamente um método construtor, a
linguagem em tempo de execução irá inicializar as propriedades com valores nulos.

Além disto, os métodos construtores - da mesma forma que métodos convencionais –
podem aceitar parâmetros de inicialização \cite{learningJava}.

Portanto, os métodos construtores são peças-chave para permitir a configuração
inicial de um objeto.

Veremos abaixo um exemplo de método construtor utilizando a linguagem PHP:

Uma vez que conhecemos os conceitos de métodos construtores, veremos a seguir  o
conceito de métodos destrutores.
\subsection{Destrutor}

A linguagem \acs{PHP} 5, traz consigo o conceito de métodos destrutores de maneira
similar a outras linguagens orientadas a objeto, como o \textit{Java}. Um método
destrutor é um método especial, que na linguagem \acs{PHP} chama-se
\textit{\_\_destruct}, em contrapartida, segundo \citeonline{learningJava}, na
linguagem \textit{Java} o método recebe o nome \textit{finalize}.

Portanto, sempre que um script terminar a sua execução ou quando um objeto for
forçadamente removido, o interpretador irá remover as referências de um objeto
da memória e o método destrutor será executado.

Na Figura \ref{fig:metodoDestrutor}, tem-se a sintaxe da definição de um
método destrutor utilizando a linguagem \acs{PHP}:

\begin{figure}[h!tb]
	\caption{Método destrutor implementado na linguagem PHP}
	\label{fig:metodoDestrutor}

	\centering
	\includegraphics[width=0.75\textwidth]{images/destruct.png}

	\centering
	\footnotesize Fonte: \fonteOAutor
\end{figure}

\FloatBarrier 	% Este comando impede que as imagens
				% flutuem a partir deste ponto no seu documento

A seguir, é apresentado em detalhes as linhas de código exibidas na Figura
\ref{fig:metodoDestrutor}:

\begin{alineas}
    \item linha 5: define-se a existência explícita de um método destrutor;
    \item linha 7: vê-se um comentário da linguagem \acs{PHP} indicando que
    neste local seria executado as funcionalidades do método destrutor, como
    por exemplo a gravação de um arquivo.
\end{alineas}

Em seguida, será definido o conceito de herança.
\subsection{Herança}
\label{heranca}

Neste capítulo serão definidos os princípios da herança, sendo que, são eles que
permitem que a programação orientada a objetos seja considerada tão poderosa.

Segundo \citeonline{programmingInObjectiveC}, através deste conceito você poderá
construir uma classe com base em uma outra classe já existente e personalizá-la
de acordo com a regra de negócio exigida pela sua aplicação.

Uma vez definido o conceito de herança, a seguir será abordardado outro conceito
da orientação a objetos, o polimorfismo.

\section{POLIMORFISMO}

Polimorfismo é a capacidade que dois ou mais objetos de uma classe-filha tem  de
responder a mensagens de diferentes formas
\cite{php5ConceitosProgramacaoEIntegracaoComBancoDeDados}.

Ou seja, polimorfismo é a possibilidade que um objeto tem de alterar o
comportamento de um objeto com base em uma classe especialista, sendo assim, 
são métodos que fornecem resultados distintos  de acordo com a subclasse 
\cite{php5ConceitosProgramacaoEIntegracaoComBancoDeDados}. Desta forma quem 
chama o método não precisa distingui-lo.

Veremos a seguir a atribuição deste conceito no processo de aceleração de um
veículo. Por conta disto, imagine o exemplo a seguir: dentre os comandos 
disponíveis em um veículo, temos o acelerador que fornece uma interface 
encapsulando a forma que compreende em como as coisas funcionam. Agora vamos 
supor que vamos acelerar dois veículos diferentes, o primeiro veículo trata-se 
de um carro popular com motor 1.0, enquanto que, o outro veículo é um 
 esportivo, e por conta disto, estamos falando de um carro com motor 2.0. Então,
ambos os veículos possuem a mesma interface de comunicação: o pedal acelerador, 
que permite ao condutor se locomover de maneira mas rápida. Note, que quando o 
condutor acelerar o carro popular este deverá ter aceleração mais lenta se 
comparado ao esportivo, na prática é disto que se trata o polimorfismo.

Voltando-se para o paradigma da orientação a objetos poderíamos ter uma classe
generalista chamada \textit{Carro} que traria uma maneira de acelerar um
veículo.  E, bom base nesta classe, criar duas outras classes especialistas, sendo que, 
poderia ser nomeadas como: \textit{CarroPopular} e \textit{CarroEsportivo}.
A seguir será apresentada a implementação destes conceitos
utilizando a linguagem de programação PHP.

\textit{Imagine o exemplo a seguir onde temos uma classe carro que possuí um
método acelerar e que existam duas outras classes que extendem a classe carro, uma 
delas é a Lamborghini a outra é “Ford Fiesta”, ambos os veículos
aceleram, entretanto a lamborgini deverá ter uma aceleração superior, sendo assim irá 
reescrever o método acelerar de acordo com as suas características, os objetos 
que chamarem esta classe não saberam como acontece mas saberam que eles
realmente  aceleram de maneira correspondente ao tipo de veículo que está sendo 
testado.}
\subsection{Encapsulamento}

Até então quando foi realizada a criação de propriedades para a nossa
classe, elas foram definidas com a visibilidade: privada, do inglês,
\textit{private}. Sendo que, sempre que definirmos as propriedades e métodos em
uma classe deve-se informar qual a  sua visibilidade. Por conseguinte, existem
três visibilidades, são elas: privada, protegida e pública. Segundo
\citeonline{javaComoProgramar}, os conceitos de visibilidade também são
chamados de modificadores de acesso.

A visibilidade privada, do inglês \textit{private}, permite acesso somente
dentro do escopo da classe que definiu a propriedade ou método. Enquanto que, a
visibilidade protegida, do inglês \textit{protected}, permite que todas as
classes  que herdam propriedades ou métodos tenham acesso dentro do escopo da classe filha.
Em contrapartida, a visibilidade pública, do inglês \textit{public}, permite que
qualquer objeto que tenha instanciado um objeto da classe que definiu uma propriedade ou
método, possa invocá-lo no caso do método, ou modificar o seu estado no caso de
uma propriedade \cite{learningJava}.

Sendo assim, por questões de segurança da aplicação é interessante fornecer
interfaces para manipulação dos estados de um objeto. Por conta disto,  sempre
que definirmos uma propriedade em uma classe será utilizada a visibilidade
privada, caso essa classe não implemente conceitos de herança posteriormente,
ou então, com visibilidade protegida a fim de permitir que as classes
especialistas  herdem as suas definições. E, é justamente este o conceito de
encapsulamento, realizar o ocultamento dos dados ou informações \cite{javaComoProgramar}.

Mas para que um cliente de um objeto – todos os objetos que estão fora do
escopo da classe - possa modificar os estados das propriedades, são criados
métodos especialistas conhecidos como: métodos \textit{getters} e \textit{setters}.

\subsubsection{Métodos getters e setters}

São métodos responsáveis por modificar e recuperar os estados de uma variável
membro de uma classe permitindo que esta função membro esteja exposta a todo o
escopo da aplicação.

A seguir na Figura \ref{fig:methodGettersAndSetters}, é apresentado um exemplo
de implementação do padrão de métodos \textit{getters and setters} na
linguagem \acs{PHP}:

\begin{figure}[h!tb]
	\caption{Métodos getters e setters na linguagem PHP}
	\label{fig:methodGettersAndSetters}

	\centering
	\includegraphics[width=0.75\textwidth]{images/method-getters-and-setters.png}

	\centering
	\footnotesize Fonte: \fonteOAutor
\end{figure}

\FloatBarrier 	% Este comando impede que as imagens
				% flutuem a partir deste ponto no seu documento

Logo abaixo, é apresentado em detalhes as linhas de código exibidas na Figura
\ref{fig:methodGettersAndSetters}:

\begin{alineas}
    \item linha 3: vê-se a declaração da classe \textit{Carro};
    \item linha 5: define-se a propriedade \textbf{\$cor};
    \item linha 7: cria-se um método responsável por recuperar o valor da
    propridade \textbf{\$cor} que recebe o nome de \textit{getCor};
    \item linha 12: implementa-se um método responsável por modificar o estado
    da propriedade cor, sendo que, o método é nomeado como \textit{setCor}.
\end{alineas}

No caso do método configurar, do inglês \textit{set}, ele permite que os
clientes de um objeto configurem novos valores para uma propriedade, realizando
antecipadamente uma ação para validar a entrada de dados.

Em contrapartida, existe o método obter, do inglês \textit{get}, que recupera o
valor de uma propriedade, sendo que, este método poderia realizar, por exemplo,
uma formatação adequada de uma propriedade.

Sendo que, o nome desses métodos (por conta de uma convenção de atribuição de
nomes) recebe o prefixo \textit{set} ou \textit{get} seguido do nome da
propriedade, tendo esta a sua primeira letra em caixa alta \cite{javaComoProgramar}.

A seguir será apresentada uma forma elegante de tratar erros em uma aplicação
orientada a objetos através de Exceções.
\subsection{Exceções}

A linguagem de programação PHP 5, introduz o conceito de exceções, e isto traz
uma grande vantagem se comparada a manipulação de erros das versões anteriores
da tecnologia, além disto, você poderá também encontrar similaridades nos
conceitos aplicados ao PHP caso tenha conhecimento de outras linguagens, tais
como: \textit{Java} e \textit{C++} \cite{phpObjectsPatternsAndPractice}.

Uma exceção indica que ocorreu uma condição não esperada ou um erro, sendo que,
isto geralmente acontece por conta de um erro de processamento, como por
exemplo: uma atribuição ou leitura de valores incorretos ou obrigatórios em
variáveis locais e propriedades durante a execuçao de um método \cite{learningJava}.

Segundo \citeonline{phpObjectsPatternsAndPractice} uma exceção é um objeto
especial instanciado a partir da classe \textit{Exception}, ou a partir de uma
classe especialista, portanto, estes objetos são projetados para criar e
reportar informações de erro.

Sendo assim, se comparado a forma tradicional de manipulação de erros, as
exceções são uma forma elegante de manipulá-los e tratá-los dentro de uma
aplicação, de acordo com o que vimos no capítulo \ref{heranca} que
tratava sobre herança, pode-se extender as funcionalidas da classe
\textit{Exception}, personalizando os seus dados e o seu comportamento \cite{phpMasterWriteCuttingEdgeCode}.

Conforme afirma \citeonline{phpMasterWriteCuttingEdgeCode}, um objeto da classe
\textit{Exception}, irá conter informações referente ao erro que ocorreu, dentre
estas informações estão:

\begin{enumerate}[a)]
    \item o nome do arquivo;
    \item a linha em que ocorreu o problema;
    \item uma mensagem;
    \item e, opcionalmente, um código de erro.
\end{enumerate}

% Esta tabela está na p.52] da referencia: phpObjectsPatternsAndPractice
\begin{table}[h!tb]
	\centering
	\setlength{\belowcaptionskip}{9pt}
	\caption[Métodos públicos da classe \textit{Exception}]{\textbf{Email local X
	Email em nuvem}}
	\begin{tabular}{| l | p{0.7\textwidth} |}

		\hline
		\textbf{Método}
		& \textbf{Descrição} \\
		\hline

        \textit{getMessage()}
        & Recupeara uma \textit{string} que foi enviada
        para o construtor.

        \\ \cline{1-2}
        \textit{getCode()}
        & Recupera o código de erro.
        \\ \cline{1-2}

        \textit{getFile()}
        & Recupera o arquivo em que a exceção foi lançada.
        \\ \cline{1-2}

        \textit{getLine()}
        & Recupera o número da linha em que a exceção ocorreu.
        \\ \cline{1-2}

        \textit{getPrevious()}
        & Recupera um objeto de uma exceção.
        \\ \cline{1-2}

        \textit{getTrace()}
        & Recupera um \textit{array} multimensional contendo as chamadas de
        métodos, incluindo: funções membro, classes, arquivos e argumentos. \\
        \cline{1-2}

        \textit{getTraceAsString()}
        & Recupera os dados de \textit{getTrace()} no formato de uma
        \textit{string}.
        \\ \cline{1-2}

        \textit{\_\_toString()}
        & Método mágico executado automaticamente quando um objeto é exibido em
        tela como uma \textit{string}. Sendo assim, retorna uma \textit{string}
        descrevendo os detalhes da exceção.
        \\
        \hline
	\end{tabular}
	\newline
	\newline
	\label{tab:excecao}
	\begin{footnotesize}
		Fonte: adaptado de \cite[p.53]{phpObjectsPatternsAndPractice}
	\end{footnotesize}
\end{table}

\FloatBarrier 	% Este comando impede que as imagens
				% flutuem a partir deste ponto no seu documento

\textit{Uma imagem que descreva como se faz para lançar uma exceção e outra
para captura-la\ldots.}
\section{INTERFACE}

As interfaces fornecem uma maneira de definir contratos entre diferentes 
classes, portanto, uma interface pode definir: a assinatura de métodos e 
valores armazenados através de constantes. Por conta disto, qualquer classe  que
quiser seguir os padrões definidos pela interface deverá assinar um contrato, 
ou seja, deverá implementar a assinatura dos métodos da mesma maneira como  fora
definido na interface \cite{programmingPhp}.

Além disto permite que possamos abstrair recursos em sistemas e facilitar a 
injeção de dependências, além de garantir a extensibilidade e padronização do 
projeto, ou seja, podemos definir que um objeto irá receber como argumento de 
um método construtor uma interface, sendo que, poderemos passar como parâmetro 
qualquer classe que implemente a interface.

Agora iremos ver um exemplo referente a definição de uma interface utilizando  a
linguagem PHP.


\chapter{BANCO DE DADOS}
\label{bancoDeDados}

Antes de abordarmos os dois bancos de dados livres mais populares:
\textit{MySQL} e \textit{PostgreSQL}, precisamos entender o que é um banco de
dados.

Segundo \citeonline{theDefinitiveGuideToMySQL5} um banco de dados pode ser uma
lista de registros que são manipulados por um programa de computador, como por exemplos o Excel, ou ainda,
pode ser também os arquivos de armazenamento de uma empresa de telecomunicações 
referente as várias chamadas que ocorreram diariamente, além disto, alguns
bancos de dados são utilizados por apenas um usuário, enquanto que alguns outros são 
acessados por vários usuários simultaneamente, enquanto que, uma base pode
ocupar poucos ou muitos kilobytes do dispositivo de armazenamento, sendo
assim, a palavra \textit{banco de dados} é utilizada para referenciar os
dados reais, o software gerenciador do banco (como por exemplo: \textit{MySQL} 
e \textit{PostgreSQL}), um cliente de conexão ao banco (são exemplos:  um script
PHP ou um programa escrito em C++), sendo assim, é comum que quando as  pessoas
falam sobre o assunto gerem confusão.

\section{MYSQL}

De acordo com \citeonline{integrandoPHP5ComMySQL}, \textit{MySQL} é um
\ac{RDBMS},  que utiliza a linguagem \ac{SQL} para manipular os seus registros, 
sendo que, é altamente utilizado em aplicações para a internet e, por
conseguinte,  seu código fonte é aberto, tendo destaque em características
como: velocidade, escalabilidade e confiabilidade, por conta disto, é adotado
pelo departamento de \ac{TI}, desenvolvedores e vendedores de software.

\section{POSTGRESQL}

Assim como, o \textit{MySQL}, o \textit{PostgreSQL} é um \acs{RDBMS} também de código
livre e surgiu na universidade da California, no projeto Berkeley, sendo que,
possuí alguns recursos de bancos empresariais, tais como: a abilidade de criar
funções agregadas e também a replicação de \textit{streaming}, certamente, estes
recursos raramente são encontrados em plataformas livres, mas, são comumente
encontradas em bancos comerciais, como: \textit{SQL Server} e \textit{IBM DB2}
e, além destas vantagens, ele pode superar bases comerciais em algumas cargas de
trabalho \cite{postgreSQLUpAndRunning}.

\chapter{DESENVOLVIMENTO FRONT-END}
\label{desenvolvimentoFrontEnd}

Geralmente quando desenvolvemos para a web, temos profissionais especialistas em
duas áreas da construção de uma aplicação, são eles: o \textit{frontend} e o
\textit{backend} \cite{artigoAvaliacaoEReducaoDoTempoDeRespostaDeSistemasWeb}.

Segundo \citeonline{artigoAvaliacaoEReducaoDoTempoDeRespostaDeSistemasWeb}, é no
\textit{frontend} que tecnologias como: a \ac{HTML}, as \ac{CSS}, o \ac{JS}
e o conteúdo multimídia são desenvolvidos.

Falaremos neste capítulo sobre essas tecnologias.

\section{HTML}

Segundo \citeonline{htmlCSSTheGoodParts} na construção de uma aplicação web ou
um website, uma das tarefas mais importantes é a criação de links entre os
documentos, possibilitando assim a navegação, o \acs{HTML} permite criarmos
estes documentos descrevendo o conteúdo que os usuários acessam ao explorar a web.

Ao acessar um website, o \abs{HTML} informa ao navegador como o seu documento
foi estruturado: qual o título do documento, se existem parágrafos, se existe
algum texto enfatizado, quais são os botões de navegação. Sendo assim, o
navegador renderiza o documento, ou seja, faz a intepretação dos
comandos e exibe para o usuário o resultado final, que neste caso, é a página
web \cite{headFirstHTMLWithCSSAndXHTML}.

\subsection{Sintaxe}

O HTML define algumas regras para a criação de documentos, dentre elas, está a
forma como os dados devem ser estruturados \cite{htmlCSSTheGoodParts}.

\section{CSS}

\ac{CSS}, do inglês \textit{Cascading Style Sheets}, é uma ferramenta
que webdesiners e também desenvolvedores utilizam em conjunto com o \ac{HTML}
para a construção de websites, onde, o \acs{CSS} proporciona que \ac{browser}
controle os aspectos visuais da página, como por exemplo: o posicionamento de elementos,
estilos de texto, cores, imagens e muito mais, além disto, existem algumas
técnicas avançadas que permitem aos autores a construção de layouts voltados a
dispositivos móveis \cite{beginningCSSCascadingStyleSheetsForWebDesign}.

Sendo assim, o \acs{CSS} permite transformar a apresentação de um ou vários
documentos \acs{HTML}

\section{JAVASCRIPT}
\subsection{Jquery}
\section{TWITTER BOOTSTRAP}

Segundo \cite{jumpStartResponsiveWebDesign}, o \acs{Twitter Bootstrap}  é uma
das mais famosas biblioteca de componentes responsivos na web.

De maneira breve, o framework permite que desenvolvedores possam criar
projetos web de maneira mais rápida e padronizada, além de permitir que
profissionais que não conheçam profundamente a linguagem de estilização 
\acs{CSS} desenvolvam interfaces agradáveis, sendo que, possuí uma manual 
extensivo que aborda exemplos de implementação, permite atingir todos os 
dispositivos com uso de layouts responsivos, pode ser usado com
pré-processadodores  tais como: \acs{LESS} e \acs{SASS} e seu código fonte é
aberto.

\section{FRAMEWORK}



\subsection{Symfony 2}
\subsection{Zend 2}