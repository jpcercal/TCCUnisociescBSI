\section{CONSTANTE}

Assim como as constantes globais utilizadas na linguagem de programação PHP 
através da função define, esta tecnologia também dispõem de uma maneira que 
permite definirmos uma constante em uma classe ou em uma interface. Da mesma 
maneira que as propriedades estáticas as constantes podem ser acessadas 
diretamente de dentro do escopo da classe utilizando-se de um operador especial 
denominado self, ou ainda, no caso da constante, ser acessada de fora do escopo 
da classe através do nome da classe  \cite{programmingPhp}.

Quando uma constante de uma classe é definida - da mesma maneira que uma 
constante global -  seu valor não poderá ser alterado no decorrer da vida útil 
da aplicação. Uma prática comumente utilizada pelos desenvolvedores de software 
é definir o nome de uma constante com caixa alta, isto permite que ela seja 
identificada rapidamente em um trecho de código. Por conseguinte, uma constante 
é um identificador que recebe apenas um valor de inicialização, por conta disto,
seu valor não poderá ser alterado durante a execução do aplicativo.

Veremos abaixo um exemplo de implementação de uma constante utilizando a 
linguagem de programação PHP: