\section{OBJETO}

Um objeto é uma estrutura dinâmica criada com base em uma classe. Sendo que, 
com base em uma classe podemos ter vários objetos, cada qual com suas 
propriedades \cite{phpProgramandoComOrientacaoAObjetos}. 

De forma breve, os objetos são as instâncias de uma classe. Sendo que, as
classes existem somente no código fonte de uma aplicação, enquanto que, as 
instâncias de uma classe existem durante a execução de um programa. Portanto, 
o software poderá criar vários objetos sob demanda tendo como base um mesmo 
modelo \cite{ios7ProgrammingFundamentalsObjectiveCXcodeAndCocoaBasics}. Deste
modo, esses objetos são criados (instanciados) através de métodos construtores 
e destruídos (eliminados) através de métodos destrutores em tempo de execução
\cite{umlEC++GuiaPraticoDeDesenvolvimentoOrientadoAObjeto}. Veremos no decorrer
deste capítulo como funcionam os métodos construtores e destrutores.

\begin{figure}[h!tb]
	\label{fig:objeto}
	
	\centering
	\caption{Criação de um Objeto na linguagem PHP}
	
	\centering
	\includegraphics[width=0.75\textwidth]{images/object.png}
	
	\centering
	\footnotesize Fonte: \fonteOAutor
\end{figure}

\FloatBarrier 	% Este comando impede que as imagens 
				% flutuem a partir deste ponto no seu documento

A seguir, será apresentada a análise do código exibido na
figura \ref{fig:objeto}: 

\begin{enumerate}[a)]
    \item \textbf{linha 1:} temos o início da execução de um bloco de código PHP;
    \item \textbf{linha 3:} criamos uma variável chamada \textbf{\$carro};
    chamamos um operador de atribuição da linguagem (representado pelo símbolo
    \textbf{=}) que irá atribuir o valor que está à direita do operador na
    variável que está à esquerda; em seguida informamos um outro operador da 
    linguagem (representado pelo símbolo \textbf{new}) que é responsável por
    criar uma referência em memória para o nosso tipo de dados que está sendo 
    criado e, por fim, dizemos que classe queremos instanciar, que neste caso, 
    chama-se Carro. Logo após, o símbolo \textbf{()} representa um método
    construtor (conceito que veremos ainda neste capítulo de Orientação a Objetos). 
    Uma informação importante é que toda instrução da linguagem PHP termina 
    com o símbolo de ponto-e-vírgula.
\end{enumerate}